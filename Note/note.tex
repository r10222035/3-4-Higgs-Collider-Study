%!TEX program = xelatex
%!TEX options=--shell-escape
\documentclass[12pt]{article}

%
\usepackage[scheme=plain]{ctex}
%
\usepackage{fontspec}
%
\usepackage[margin = 1in]{geometry}

%
\usepackage[dvipsnames]{xcolor}
\usepackage[many]{tcolorbox}

%
\usepackage{amsmath}
\usepackage{amssymb}
\usepackage{amsthm}
%
\usepackage{tensor}
%
\usepackage{slashed}
\usepackage{physics}
\usepackage{simpler-wick}

%
\usepackage[version=4]{mhchem}

%
\usepackage{mathtools}

%
\usepackage{bm}
\newcommand{\dbar}{\dif\hspace*{-0.18em}\bar{}\hspace*{0.2em}}
\DeclareMathAlphabet\mathbfcal{OMS}{cmsy}{b}{n}
%\usepackage{bbold}
\newcommand*{\dif}{\mathop{}\!\mathrm{d}}
\newcommand*{\euler}{\mathrm{e}}
\newcommand*{\imagi}{\mathrm{i}}

\renewcommand{\vec}[1]{\boldsymbol{\mathbf{#1}}}

\usepackage{caption}

\usepackage{enumitem}

%
\usepackage{mathrsfs}
\usepackage{dsfont}

%
\usepackage{hyperref}
\hypersetup{
    colorlinks=true,
    linkcolor=violet,
    filecolor=blue,      
    urlcolor=blue,
    citecolor=cyan,
}

%
\usepackage{graphicx}
%
\graphicspath{{image/}}


%
\usepackage{indentfirst}
%
\setlength{\parindent}{2em}
\linespread{1.25}

% 
% \setmainfont{Times New Roman}

\title{Note}
\author{Feng-Yang Hsieh}
\date{}

\begin{document}
\maketitle
\section{Signal and background}% (fold)
\label{sec:signal_and_background}
	Signal events are 3 Higgs events in BSM. I use MadGraph to generate by following commands:
	\begin{verbatim}
		import model cxSM_VLF_EFT
		generate g g > h h h
	\end{verbatim}
	then pass the events through the Pythia and Delphes. In Pythia, I set the branching ratio of Higgs so that it only decays to b quarks.

	Background events are 6b events in SM. I use MadGraph to generate by following commands:
	\begin{verbatim}
		generate p p > b b b b~ b~ b~
	\end{verbatim}
	then pass the events through the Pythia and Delphes.

% section signal_and_background (end)
\section{Transverse momentum and pseudorapidity distribution}% (fold)
\label{sec:transverse_momentum_and_pseudorapidity_distribution}
	In each event there are 6 b quarks. I order these b quarks by transverse momentum $p_\text{T}$, then plot $p_\text{T} $ and pseudorapidity $\eta$ distributions.

	Figure \ref{fig:background_pt_eta_distribution_in_parton_level} is the  $p_\text{T}$ and  $\eta$ distributions of b quarks for background.	
	\begin{figure}[htpb]
		\centering
		\includegraphics[width=0.8\textwidth]{ppbbbbbb_sm_PT_Eta_order_by_PT.png}
		\caption{$p_\text{T}$ and $\eta$ distributions of b partons for background events. They are ordered by $p_\text{T}$.}
		\label{fig:background_pt_eta_distribution_in_parton_level}
	\end{figure}

	Figure \ref{fig:signal_pt_eta_distribution_in_parton_level} is the $p_\text{T}$ and $\eta$ distributions of b quarks for signal.

	\begin{figure}[htpb]
		\centering
		\includegraphics[width=0.8\textwidth]{gghhh_bsm_PT_Eta_order_by_PT.png}
		\caption{$p_\text{T}$ and $\eta$ distributions of b partons for signal events. They are ordered by $p_\text{T}$.}
		\label{fig:signal_pt_eta_distribution_in_parton_level}
	\end{figure}
% section transverse_momentum_and_pseudorapidity_distribution (end)
\section{Delta R and transverse momentum distribution}% (fold)
\label{sec:delta_r_and_transverse_momentum_distribution}
	For signal, I plot the $\Delta R(\text{b,b})$ distribution of two b decays from the same Higgs. Then I plot $\Delta R(\text{b,b})$ against $p_\text{T}(\text{b,b})$. The result is in Figure \ref{fig:dR(b,b)_and_pT(b,b)}. As can be seen from the plot, $\Delta R$ get smaller at higher $p_\text{T}(\text{b,b})$. 
	\begin{figure}[htpb]
		\centering
		\includegraphics[width=1\textwidth]{gghhh_bsm_bb_Delta_R.png}
		\caption{$\Delta R(\text{b,b})$ distribution and $\Delta R(\text{b,b})$ against $p_\text{T}(\text{b,b})$ plot}
		\label{fig:dR(b,b)_and_pT(b,b)}
	\end{figure}
% section delta_r_and_transverse_momentum_distribution (end)
\section{Cutflow table}% (fold)
\label{sec:cutflow_table}
I apply the $\eta$ and $p_\text{T}$ cuts sequentially and count how many event can pass the cuts. The results are in Table \ref{tab:cutflow_table_background} and Table \ref{tab:cutflow_table_signal}.
	\begin{table}[htpb]
		\centering
		\caption{1000 background events. Those background events is generated in $\sqrt{s} = 13 \text{ TeV}$ with the cut b $p_\text{T} > 10 \text{ GeV}$ in MadGraph.
}
		\label{tab:cutflow_table_background}
		\begin{tabular}{cc}
			Cut & number of events pass \\
			\hline
			6 b-partons $\abs{\eta}<2.5$ & 330 \\
			6 b-partons $p_\text{T} < 25 \text{ GeV}$ & 11 \\
			4 b-partons $p_\text{T} < 40 \text{ GeV}$ & 8                            
		\end{tabular}	
	\end{table}
	
	\begin{table}[htpb]
		\centering
		\caption{1000 signal events.}
		\label{tab:cutflow_table_signal}
		\begin{tabular}{cc}
			Cut & number of events pass \\ 
			\hline
			6 b-partons $\abs{\eta}<2.5$ & 732 \\
			6 b-partons $p_\text{T} < 25 \text{ GeV}$ & 534 \\
			4 b-partons $p_\text{T} < 40 \text{ GeV}$ & 498                            
		\end{tabular}	
	\end{table}
% section cutflow_table (end)

\section{Construction of bb-pairs}% (fold)
\label{sec:construction_of_bb_pairs}
	To construct the bb-pairs from the same Higgs, the typical strategy is to try all combinatoric and find the minimal mass difference between all pairs, i.e. minimize this
	\[
		\chi^2 = [M(\text{b}_1\text{b}_2) - M(\text{b}_3 \text{b}_4)]^2 + [M(\text{b}_1\text{b}_2) - M(\text{b}_5 \text{b}_6)]^2 +[M(\text{b}_3\text{b}_4) - M(\text{b}_5 \text{b}_6)]^2
	\] 
	where $M(\text{b}_i \text{b}_j)$ means the invariant mass of b-parton $i$ and b-parton $j$.

	The other strategy is to minimize mass difference to the SM Higgs mass
	\[
		\chi^2 = [M(\text{b}_1\text{b}_2) - M_\text{H}]^2 + [M(\text{b}_3\text{b}_4) - M_\text{H}]^2 +[M(\text{b}_5\text{b}_6) - M_\text{H}]^2
	\] 

	I implement these two methods to construct bb-pairs and test on the 10,000 signal events in parton level. The result is as follows:
	\begin{itemize}
		\item Method 1 (mass difference between all pairs): accuracy = 0.863
		\item Method 2 (mass difference to the SM Higgs mass): accuracy = 0.875
	\end{itemize}


	Method 2 is better for identifying the true Higgs pair.
% section construction_of_bb_pairs (end)

\section{Absolute cross section}% (fold)
\label{sec:absolute_cross_section}
	Absolute cross section is defined as follow
	\[
		\sigma_\text{abs} = \sigma \times \frac{\text{number of events pass the cut}}{\text{number of events}}
	\] 

	I have generated background events with different cuts in MadGraph, the detailed information of cuts is in Table \ref{tab:cut_of_different_run}. I calculate their absolute cross section and check if it is the same or not. The result is in Table \ref{tab:absolute_cross_section}.

	\begin{table}[htpb]
		\centering
		\caption{The cuts applied on the different run. Where $p_\text{T}$ means the minimum transverse momentum of b and $\abs{\eta}$ means the range of b (-1 means no restriction).}
		\label{tab:cut_of_different_run}
		\begin{tabular}{ccc}
			No.& $p_\text{T}$ (GeV) & $\abs{\eta}$  \\
			1 &      0       &           -1      \\
			2 &      10       &     5          \\
			3 &      10       &     3 \\
			4 &      20       &     3 \\
			5 &      0       &     -1 \\
			6 &      10       &     -1 
		\end{tabular}	
	\end{table}


	\begin{table}[htpb]
		\centering
		\caption{Absolute cross section. Where the selection cut is $p_\text{T}> \text{20 GeV}, \abs{\eta} < 3$.}
		\label{tab:absolute_cross_section}
		\begin{tabular}{llllll}
			No.& total event & cross section & events pass selection   & absolute cross section \\
			1& 339 & 2731.9362 & 0 & 0.0    \\
			2& 4761 & 71.316651 & 190 & 2.846   \\
			3& 9054 & 35.225852 & 694 & 2.700 \\
			4& 4207 & 2.7698473 & 4183 & 2.754 \\
			5& 1000 & 2531.9304 & 0 & 0.0 \\
			6& 1000 & 70.158431 & 27 & 1.894
		\end{tabular}	
	\end{table}

	From the Table \ref{tab:absolute_cross_section}, if the number of event is large enough the absolute cross section will be similar.
% section absolute_cross_section (end)
\section{The problem for generating \texorpdfstring{pp $\to$ 6b}{pp to 6b} events}% (fold)
\label{sec:the_problem_for_generating_pp_6b_events}
	\href{https://answers.launchpad.net/mg5amcnlo/+question/701314}{Failed to generate the requested number of events}
% section the_problem_for_generating_pp_6b_events (end)	
\section{\texorpdfstring{pp $\to $ 4b}{pp to 4b}}% (fold)
\label{sec:pp_4b}
	I use MadGraph to generate 4b events by following commands:
	\begin{verbatim}
		generate p p > b b b~ b~ 
	\end{verbatim}
	In MadGraph, I apply the cuts: $p_\text{T}>\text{25 GeV}, \abs{\eta}<2.5$ for b. Pass those events through Pythia, then check how many events there are 6 b-hadrons final state.

	In 100,000 events, there are 6,916 events having greater than or equal to 6 b-hadrons. The distribution of number of b-hadrons is in Figure \ref{fig:number_of_b_hadron}.
	\begin{figure}[htpb]
		\centering
		\includegraphics[width=0.8\textwidth]{number_of_b-hadrons.png}
		\caption{Number of b-hadrons in Pythia final state.}
		\label{fig:number_of_b_hadron}
	\end{figure}

	Pass events through Delphes. In Delphes the b-tagging efficiency is set to 1.

	Figure \ref{fig:number_of_jet} is the number of jets and number of b-jets distributions. In 100,000 events, there are 173 events having greater than or equal to 6 b-jets.
	\begin{figure}[htpb]
		\centering
		\includegraphics[width=1\textwidth]{number_of_b-jets.png}
		\caption{Number of jets and number of b-jets.}
		\label{fig:number_of_jet}
	\end{figure}
% section pp_4b (end)

\section{Comparison for \texorpdfstring{pp $\to$ 4b}{pp to 4b} and \texorpdfstring{pp $\to$ 6b}{pp to 6b} event}% (fold)
\label{sec:comparison_for_pp_4b_and_pp_6b_event}
	To compare pp->4b and pp->6b events, I plot the $p_\text{T}$, $\eta$ and total invariant mass of 6 b-jets for both.  
	
	The number of events of 6 b-jets for pp->4b is 2051 and for pp->6b is 1408. I scaled the number of events for pp->4b to be the same as pp->6b.

	Figure \ref{fig:pt_eta_distribution_of_b_jets} is $p_\text{T}$ and $\eta$ distributions. Figure \ref{fig:total_invariant_mass_of_b_jets} is total invariant mass of 6 b-jets distributions. Their distributions look similar.

	\begin{figure}[htpb]
		\centering
		\includegraphics[width=0.7\textwidth]{6_b-jets_PT_Eta_order_by_PT.png}
		\caption{$p_\text{T}$ and $\eta$ distributions of b-jets for pp->4b and pp->6b events. They are ordered by $p_\text{T}$.}
		\label{fig:pt_eta_distribution_of_b_jets}
	\end{figure}
	\begin{figure}[htpb]
		\centering
		\includegraphics[width=0.7\textwidth]{6_b-jets_total_invariant_mass.png}
		\caption{The distribution of total invariant mass of 6 b-jets for pp->4b and pp->6b events.}
		\label{fig:total_invariant_mass_of_b_jets}
	\end{figure}
	
% section comparison_for_pp_4b_and_pp_6b_event (end)	
\section{Cutflow table for b-jets}% (fold)
\label{sec:cutflow_table_for_b_jets}
	I apply following cut sequentially and count how many events can pass these cuts. Table \ref{tab:cutflow_table_pp4b_bjet} is the result for pp->4b events. Table \ref{tab:cutflow_table_pp6b_bjet} is the result for pp->6b events. Table \ref{tab:cutflow_table_signal_bjet} is the result for signal events.

	\begin{itemize}
		\item Cut 1: There are greater than or equal to 6 b-jets.
		\item Cut 2: There are greater than or equal to 6 b-jets satisfy $\abs{\eta}<2.5$.
		\item Cut 3: There are greater than or equal to 6 b-jets satisfy $p_\text{T}>\text{25 GeV}$.
		\item Cut 4: There are greater than or equal to 4 b-jets satisfy $p_\text{T}>\text{40 GeV}$.
	\end{itemize}

	\begin{table}[htpb]
		\centering
		\caption{1,000,000 pp->4b events. Those events are generated in $\sqrt{s} = 14 \text{ TeV}$ with the cuts: $p_\text{T} >\text{25 GeV}, \abs{\eta}<2.5$ for b in MadGraph.
}
		\label{tab:cutflow_table_pp4b_bjet}
		\begin{tabular}{cc}
			Cut & number of event pass \\
			\hline
			1 & 1783 \\
			2 & 1565 \\
			3 & 1059 \\
			4 & 735 
		\end{tabular}	
	\end{table}

	\begin{table}[htpb]
		\centering
		\caption{10,000 pp->6b events. Those events are generated in $\sqrt{s} = 14 \text{ TeV}$ with the cuts: $p_\text{T} >\text{25 GeV}, \abs{\eta}<2.5$ for b in MadGraph.
}
		\label{tab:cutflow_table_pp6b_bjet}
		\begin{tabular}{cc}
			Cut & number of event pass \\
			\hline
			1 & 1408 \\
			2 & 1322 \\
			3 & 1052 \\
			4 & 822 
		\end{tabular}	
	\end{table}

	\begin{table}[htpb]
		\centering
		\caption{100,000 signal events. Those events are generated in $\sqrt{s} = 14 \text{ TeV}$.}
		\label{tab:cutflow_table_signal_bjet}
		\begin{tabular}{cc}
			Cut & number of event pass \\
			\hline
			1 & 21,814 \\
			2 & 21,254 \\
			3 & 17,130 \\
			4 & 14,142 
		\end{tabular}	
	\end{table}
% section cutflow_table_for_b_jets (end)

\section{Signal}% (fold)
\label{sec:signal}
	Generate resonant channel gg->h3, h3->h2h, h2->hh in MadGraph by following commands:
	\begin{verbatim}
		import model cxSM_VLF_EFT
		generate g g > h3, (h3 > h2 h, h2 > h h)
	\end{verbatim}
	then check this channel is dominated in signal (gg->3h) or not.

	I generate this channel and signal in $\sqrt{s} = \text{14 TeV}$, the cross sections are 2.094 pb and 4.067 pb, respectively.

	Figure \ref{fig:signal_invariant_mass_of_6bjet} is the total invariant mass of 6 b-jets for signal events. Figure \ref{fig:resonant_channel_invariant_mass_of_6bjet} is the total invariant mass of 6 b-jets for this resonant channel. From the results, there is a peak around $\text{400 GeV}$, because the mass of $h_3$ is $m_3 = \text{420 GeV}$.

	\begin{figure}[htpb]
		\centering
		\includegraphics[width=0.8\textwidth]{signal_6_b-jets_total_invariant_mass.png}
		\caption{Total invariant mass of 6 b-jets for signal events.}
		\label{fig:signal_invariant_mass_of_6bjet}
	\end{figure}

	\begin{figure}[htpb]
		\centering
		\includegraphics[width=0.8\textwidth]{signal_resonant_6_b-jets_total_invariant_mass.png}
		\caption{Total invariant mass of 6 b-jets for resonant channel.}
		\label{fig:resonant_channel_invariant_mass_of_6bjet}
	\end{figure}

	A bump around $\text{600 GeV}$ in Figure \ref{fig:signal_invariant_mass_of_6bjet}? In run card, some parameters are not set correctly. After setting all parameters correctly and regenerating these event, the results are in Figure \ref{fig:invariant_mass_of_6bjet_correct}. There is no bump around $\text{600 GeV}$.

	\begin{figure}[htpb]
		\centering
		\includegraphics[width=0.49\textwidth]{signal_6_b-jets_total_invariant_mass_correct.png}
		\includegraphics[width=0.49\textwidth]{signal_resonant_6_b-jets_total_invariant_mass_correct.png}
		\caption{Total invariant mass of 6 b-jets for signal events and resonant channel (correct parameters).}
		\label{fig:invariant_mass_of_6bjet_correct}
	\end{figure}



% section signal (end)
	
\section{The cross section in MadGraph}% (fold)
\label{sec:the_cross_section_in_madgraph}
Table \ref{tab:Cross_section_from_MadGraph_and_paper} is the cross sections calculated by MadGraph and in paper. They are different because in MadGraph we only consider LO. But in the paper, the numbers are quoted from the LHC Higgs Cross Section Working Group, and they have considered up to NNLO. The ``k factor'' is around 2.5, which accounts for the values in gg>h3.
	\begin{table}[htpb]
		\centering
		\caption{Cross section from MadGraph and paper}
		\label{tab:Cross_section_from_MadGraph_and_paper}
		\begin{tabular}{ccc}
			Process & $\sigma$  MadGraph (fb)  & $\sigma$ in paper (fb)\\
			\hline
			g g > h3 & 21 & 55 \\
			gg > h3 > h h h & 4.0 &  38.2 \\
			\hline
		\end{tabular}	
	\end{table}

	The problem of cross section: If we use the default run card to generate the decay process, the value of cross section will be problematic. 

	Solution:
	
	In madevent, use command \verb+compute_widths+ to compute decay widths of $h_2,h_3$, then replace the \verb+run_card.dat+ by \verb+run_card_default.dat+.

	Regenerate the signal and resonant event, the cross sections are $\text{11.1 fb}$ and $\text{10.39 fb}$, respectively. This channel is indeed dominated in the signal.
% section the_cross_section_in_madgraph (end)	
\section{Comparision for \texorpdfstring{pp $\to $ hhh}{pp to hhh} and gg \texorpdfstring{$\to $}{to} hhh}% (fold)
\label{sec:comparision_for_pp_to_hhh_and_gg_to_hhh}
	Generate pp->hhh in MadGraph by following commands:
	\begin{verbatim}
		import model cxSM_VLF_EFT
		define p = p b b~
		generate p p > h h h QCD<=8	
	\end{verbatim}

	Generate gg->hhh in MadGraph by following commands:
	\begin{verbatim}
		import model cxSM_VLF_EFT
		generate g g > h h h	
	\end{verbatim}

	These events are generated in $\text{14 TeV}$ and the sample size are 100k. The cross section for pp->hhh and gg->hhh are $\text{11.22 fb}$ and $\text{11.11 fb}$, respectively. They only differ by 1\%.
% section comparision_for_pp_to_hhh_and_gg_to_hhh (end)		
\section{Comparision for \texorpdfstring{gg $\to $ hhh}{gg to hhh} and resonant channel}% (fold)
\label{sec:comparision_for_gg_hhh_and_resonant_channel}
	To compare $\text{gg $\to $ hhh}$ and resonant channel, I plot the $p_\text{T}$, $\eta$, and total invariant mass of 6 b-jets for both.

	These events are generated in $\text{14 TeV}$ and the sample size are 100k. The cross section for gg->hhh and resonant channel are $\text{11.11 fb}$ and $\text{10.38 fb}$, respectively.

	The number of events of 6 b-jets for gg->hhh is 21,814 and for the resonant channel is 21,475. The numbers are very close. I scaled the number of events for gg->3h to be the same as the resonant channel.

	Figure \ref{fig:signal_resonant_pt_eta_distribution_of_b_jets} is $p_\text{T}$ and $\eta$ distributions. Figure \ref{fig:signal_resonant_total_invariant_mass_of_b_jets} is total invariant mass of 6 b-jets distributions. Their distributions look similar.


	\begin{figure}[htpb]
		\centering
		\includegraphics[width=0.7\textwidth]{Signal_PT_Eta_order_by_PT.png}
		\caption{$p_\text{T}$ and $\eta$ distributions of b-jets for gg->3h and resonant channel. They are ordered by $p_\text{T}$.}
		\label{fig:signal_resonant_pt_eta_distribution_of_b_jets}
	\end{figure}
	\begin{figure}[htpb]
		\centering
		\includegraphics[width=0.7\textwidth]{Signal_and_resonant_total_invariant_mass.png}
		\caption{The distribution of total invariant mass of 6 b-jets for gg->3h and resonant channel.}
		\label{fig:signal_resonant_total_invariant_mass_of_b_jets}
	\end{figure}

	I apply the cuts same in Sec.\ref{sec:cutflow_table_for_b_jets} and count how many events can pass these cuts. Table \ref{tab:cutflow_table_signal_and_resonant_bjet} is the result. Those results for both events are similar.
	\begin{table}[htpb]
		\centering
		\caption{Number of events pass the selection cuts. Total number of events for gg->hhh and resonant channel both are 100,000.}
		\label{tab:cutflow_table_signal_and_resonant_bjet}
		\begin{tabular}{ccc}
			Cut &  gg->3h & resonant channel \\
			\hline
			1 & 21,814 & 21,475\\
			2 & 21,254 & 20,898\\
			3 & 17,130 & 16,828\\
			4 & 14,142 & 13,730
		\end{tabular}	
	\end{table}

% section comparision_for_gg_hhh_and_resonant_channel (end)		
\section{Branching ratios and decay widths}% (fold)
\label{sec:branching_ratios_and_decay_widths}
	The branching ratios and decay widths are calculated by MadGraph and Mathematica notebook. Mathematica notebook does not include uu dd ss ee decay modes. Since the contribution of these modes is very small. 

	Table \ref{tab:MG_decay_widths_BR} is calculated by MadGraph. Table \ref{tab:MA_decay_widths_BR} is calculated by Mathematica notebook.
	\begin{table}[htpb]
		\centering
		\caption{Decay widths and branching ratios calulated by MadGraph at BP1.}
		\label{tab:MG_decay_widths_BR}
		\begin{tabular}{lll}
								  & BR        & Width (GeV) \\
			$\text{h}_2\to\text{h}_1\text{h}_1$ & 0.7506    & 0.085616    \\
			$\text{h}_2\to\text{WW}$   & 0.1734    & 0.019782    \\
			$\text{h}_2\to\text{ZZ}$   & 0.07548   & 0.0086097   \\
			$\text{h}_2\to\text{bb}$   & 0.0004724 & 5.3887e-05  \\
			$\text{h}_2\to\text{cc}$   & 3.455e-05 & 3.9407e-06  \\
			$\text{h}_2\to\tau\tau$   & 2.254e-05 & 2.5714e-06  \\
			$\text{h}_2\to\text{ss}$   & 2.185e-07 & 2.4927e-08  \\
			$\text{h}_2\to\text{gg}$   & 1.561e-07 & 1.7811e-08  \\
			$\text{h}_2\to\mu\mu$   & 7.972e-08 & 9.0933e-09  \\
			$\text{h}_2\to\gamma\gamma$   & 6.495e-09 & 7.409e-10   \\
			$\text{h}_2\to\gamma\text{Z}$   & 9.447e-10 & 1.0776e-10  \\
			$\text{h}_2\to\text{dd}$   & 5.441e-10 & 6.207e-11   \\
			$\text{h}_2\to\text{uu}$   & 1.393e-10 & 1.5889e-11  \\
			$\text{h}_2\to\text{ee}$   & 1.865e-12 & 2.1269e-13  \\
			\\
			\\
		\end{tabular}	
		\begin{tabular}{lll}
								  & BR        & Width (GeV) \\
			$\text{h}_3\to\text{h}_1\text{h}_2$ & 0.6795    & 0.8414      \\
			$\text{h}_3\to\text{h}_1\text{h}_1$ & 0.1711    & 0.21183     \\
			$\text{h}_3\to\text{WW}$   & 0.08756   & 0.10842     \\
			$\text{h}_3\to\text{ZZ}$   & 0.04102   & 0.050792    \\
			$\text{h}_3\to\text{tt}$   & 0.02065   & 0.025569    \\
			$\text{h}_3\to\text{gg}$   & 8.662e-05 & 0.00010725  \\
			$\text{h}_3\to\text{bb}$   & 8.158e-05 & 0.00010102  \\
			$\text{h}_3\to\text{cc}$   & 5.961e-06 & 7.3813e-06  \\
			$\text{h}_3\to\tau\tau$   & 3.89e-06  & 4.8167e-06  \\
			$\text{h}_3\to\gamma\gamma$   & 3.852e-07 & 4.7697e-07  \\
			$\text{h}_3\to\gamma\text{Z}$   & 4.937e-08 & 6.1131e-08  \\
			$\text{h}_3\to\text{ss}$   & 3.77e-08  & 4.6686e-08  \\
			$\text{h}_3\to\mu\mu$   & 1.375e-08 & 1.7031e-08  \\
			$\text{h}_3\to\text{dd}$   & 9.389e-11 & 1.1625e-10  \\
			$\text{h}_3\to\text{uu}$   & 2.403e-11 & 2.976e-11   \\
			$\text{h}_3\to\text{ee}$   & 3.217e-13 & 3.9835e-13 
		\end{tabular}
	\end{table}
	\begin{table}[htpb]
		\centering
		\caption{Decay widths and branching ratios calulated by Mathematica at BP1.}
		\label{tab:MA_decay_widths_BR}
		\begin{tabular}{lll}
									  & BR        & Width (GeV) \\
			$\text{h}_2\to\text{h}_1\text{h}_1$     & 0.8321    & 0.1407      \\
			$\text{h}_2\to\text{WW}$       & 0.1166    & 0.019721    \\
			$\text{h}_2\to\text{ZZ}$       & 0.05108   & 0.0086386   \\
			$\text{h}_2\to\text{gg}$       & 0.0001204 & 2.0359e-05  \\
			$\text{h}_2\to\text{bb}$       & 9.91e-05  & 1.6758e-05  \\
			$\text{h}_2\to\tau\tau$       & 1.526e-05 & 2.5802e-06  \\
			$\text{h}_2\to\text{cc}$       & 4.59e-06  & 7.7613e-07  \\
			$\text{h}_2\to\gamma\gamma$       & 3.161e-06 & 5.346e-07   \\
			$\text{h}_2\to\gamma\text{Z}$       & 4.822e-07 & 8.154e-08   \\
			$\text{h}_2\to\mu\mu$μμ       & 5.43e-08  & 9.1826e-09  \\
			$\text{h}_2\to\text{tt}$       & 4.355e-19 & 7.3649e-20  \\
			\\
		\end{tabular}		
		\begin{tabular}{lll}
								  & BR        & Width (GeV) \\
			$\text{h}_3\to\text{h}_1\text{h}_2$ & 0.8219    & 2.158       \\
			$\text{h}_3\to\text{h}_1\text{h}_1$ & 0.1049    & 0.27543     \\
			$\text{h}_3\to\text{WW}$   & 0.04131   & 0.10846     \\
			$\text{h}_3\to\text{ZZ}$   & 0.01941   & 0.050962    \\
			$\text{h}_3\to\text{tt}$   & 0.01242   & 0.032613    \\
			$\text{h}_3\to\text{gg}$   & 6.364e-05 & 0.00016711  \\
			$\text{h}_3\to\text{bb}$   & 1.195e-05 & 3.1386e-05  \\
			$\text{h}_3\to\tau\tau$   & 1.841e-06 & 4.8326e-06  \\
			$\text{h}_3\to\text{cc}$   & 5.536e-07 & 1.4536e-06  \\
			$\text{h}_3\to\gamma\gamma$   & 1.823e-07 & 4.7857e-07  \\
			$\text{h}_3\to\gamma\text{Z}$   & 2.334e-08 & 6.1277e-08  \\
			$\text{h}_3\to\mu\mu$   & 6.55e-09  & 1.7198e-08  
		\end{tabular}
	\end{table}

	The exact values from both are slightly different because MG can not correctly expand the parameter $\epsilon$ in the model.

	Set the $\epsilon$ to $0.001$ and calculate again. The results are in Table \ref{tab:MG_decay_widths_BR_0001} and Table \ref{tab:MA_decay_widths_BR_0001}.

	\begin{table}[htpb]
		\centering
		\caption{Decay widths and branching ratios calulated by MadGraph at BP1 with $\epsilon=0.001$.}
		\label{tab:MG_decay_widths_BR_0001}
		\begin{tabular}{lll}
								  & BR        & Width (GeV) \\
			$\text{h}_2\to\text{h}_1\text{h}_1$ & 0.8311    & 1.4041e-05  \\
			$\text{h}_2\to\text{WW}$   & 0.1174    & 1.9835e-06  \\
			$\text{h}_2\to\text{ZZ}$   & 0.0511    & 8.6329e-07  \\
			$\text{h}_2\to\text{bb}$   & 0.0003198 & 5.4032e-09  \\
			$\text{h}_2\to\text{cc}$   & 2.339e-05 & 3.9513e-10  \\
			$\text{h}_2\to\tau\tau$   & 1.526e-05 & 2.5783e-10  \\
			$\text{h}_2\to\text{ss}$   & 1.479e-07 & 2.4994e-12  \\
			$\text{h}_2\to\mu\mu$μμ   & 5.397e-08 & 9.1178e-13  \\
			$\text{h}_2\to\text{dd}$   & 3.684e-10 & 6.2237e-15  \\
			$\text{h}_2\to\text{uu}$   & 9.431e-11 & 1.5932e-15  \\
			$\text{h}_2\to\text{gg}$   & 1.056e-11 & 1.7841e-16  \\
			$\text{h}_2\to\text{ee}$   & 1.262e-12 & 2.1326e-17  \\
			$\text{h}_2\to\gamma\gamma$   & 4.393e-13 & 7.4213e-18  \\
			$\text{h}_2\to\gamma\text{Z}$   & 6.389e-14 & 1.0794e-18 \\
			\\
			\\
		\end{tabular}
		\begin{tabular}{lll}
								  & BR        & Width (GeV) \\
			$\text{h}_3\to\text{h}_1\text{h}_2$ & 1.0       & 2.1493      \\
			$\text{h}_3\to\text{h}_1\text{h}_1$ & 1.279e-05 & 2.748e-05   \\
			$\text{h}_3\to\text{WW}$   & 5.058e-06 & 1.0871e-05  \\
			$\text{h}_3\to\text{ZZ}$   & 2.369e-06 & 5.0929e-06  \\
			$\text{h}_3\to\text{tt}$   & 1.193e-06 & 2.5638e-06  \\
			$\text{h}_3\to\text{gg}$   & 5.007e-09 & 1.0761e-08  \\
			$\text{h}_3\to\text{bb}$   & 4.713e-09 & 1.0129e-08  \\
			$\text{h}_3\to\text{cc}$   & 3.443e-10 & 7.4011e-10  \\
			$\text{h}_3\to\tau\tau$   & 2.247e-10 & 4.8297e-10  \\
			$\text{h}_3\to\gamma\gamma$   & 2.227e-11 & 4.7857e-11  \\
			$\text{h}_3\to\gamma\text{Z}$   & 2.854e-12 & 6.1335e-12  \\
			$\text{h}_3\to\text{ss}$   & 2.178e-12 & 4.6812e-12  \\
			$\text{h}_3\to\mu\mu$μμ   & 7.945e-13 & 1.7077e-12  \\
			$\text{h}_3\to\text{dd}$   & 5.423e-15 & 1.1657e-14  \\
			$\text{h}_3\to\text{uu}$   & 1.388e-15 & 2.984e-15   \\
			$\text{h}_3\to\text{ee}$   & 1.858e-17 & 3.9942e-17 
		\end{tabular}	
	\end{table}

	\begin{table}[htpb]
		\centering
		\caption{Decay widths and branching ratios calulated by Mathematica at BP1 with $\epsilon=0.001$.}
		\label{tab:MA_decay_widths_BR_0001}
		\begin{tabular}{lll}
								  & BR        & Width (GeV) \\
			$\text{h}_2\to\text{h}_1\text{h}_1$ & 0.8321    & 1.407e-05   \\
			$\text{h}_2\to\text{WW}$   & 0.1166    & 1.9721e-06  \\
			$\text{h}_2\to\text{ZZ}$   & 0.05108   & 8.6386e-07  \\
			$\text{h}_2\to\text{gg}$   & 0.0001204 & 2.0359e-09  \\
			$\text{h}_2\to\text{bb}$   & 9.91e-05  & 1.6758e-09  \\
			$\text{h}_2\to\tau\tau$   & 1.526e-05 & 2.5802e-10  \\
			$\text{h}_2\to\text{cc}$   & 4.59e-06  & 7.7613e-11  \\
			$\text{h}_2\to\gamma\gamma$   & 3.161e-06 & 5.346e-11   \\
			$\text{h}_2\to\gamma\text{Z}$  & 4.822e-07 & 8.154e-12   \\
			$\text{h}_2\to\mu\mu$   & 5.43e-08  & 9.1826e-13  \\
			$\text{h}_2\to\text{tt}$   & 4.355e-15 & 7.3649e-20  \\
			\\
		\end{tabular}
		\begin{tabular}{lll}
								  & BR        & Width (GeV) \\
			$\text{h}_3\to\text{h}_1\text{h}_2$ & 1.0       & 2.158       \\
			$\text{h}_3\to\text{h}_1\text{h}_1$ & 1.276e-05 & 2.7543e-05  \\
			$\text{h}_3\to\text{WW}$   & 5.026e-06 & 1.0846e-05  \\
			$\text{h}_3\to\text{ZZ}$   & 2.361e-06 & 5.0962e-06  \\
			$\text{h}_3\to\text{tt}$   & 1.511e-06 & 3.2613e-06  \\
			$\text{h}_3\to\text{gg}$   & 7.743e-09 & 1.6711e-08  \\
			$\text{h}_3\to\text{bb}$   & 1.454e-09 & 3.1386e-09  \\
			$\text{h}_3\to\tau\tau$   & 2.239e-10 & 4.8326e-10  \\
			$\text{h}_3\to\text{cc}$   & 6.736e-11 & 1.4536e-10  \\
			$\text{h}_3\to\gamma\gamma$   & 2.218e-11 & 4.7857e-11  \\
			$\text{h}_3\to\gamma\text{Z}$   & 2.839e-12 & 6.1277e-12  \\
			$\text{h}_3\to\mu\mu$   & 7.969e-13 & 1.7198e-12  
		\end{tabular}
	\end{table}
	This time, the decay widths and branching ratios calculated by MG and Mathematica notebook become close.

	For color channels, they are still doesn't look the same. The reason is that in MadGraph it only calculates to LO, but the numbers in Mathematica notebook are calculated to NLO.
% section branching_ratios_and_decay_widths (end)	
\section{SPANet}% (fold)
\label{sec:spanet}
	Code: \href{https://github.com/Alexanders101/SPANet}{Symmetry Preserving Attention Networks}

	Symmetry Preserving Attention NETworks (Spa-Net) is used to do the jet assignment task. The jet assignment task is the identification of the original particle which leads to a reconstructed jet.

	\subsection{Prepare training data}% (fold)
	\label{sub:prepare_training_data}
		\begin{enumerate}
			\item Defining the event topology in \verb+.ini+ file. The structure of the \verb+.ini+ file follows this format:

			\begin{verbatim}
				[SOURCE]
				FEATURE_1 = FEATURE_OPTION
				FEATURE_2 = FEATURE_OPTION
				FEATURE_3 = FEATURE_OPTION
				...

				[EVENT]
				particles = (PARTICLE_1, PARTICLE_2, ...)
				permutations = EVENT_SYMMETRY_GROUP

				[PARTICLE_1]
				jets = (JET_1, JET_2, ...)
				permutations = JET_SYMMETRY_GROUP

				[PARTICLE_2]
				jets = (JET_1, JET_2, ...)
				permutations = JET_SYMMETRY_GROUP

				...
			\end{verbatim}

			\item Create training dataset in HDF5 format.

			\item Write option-file in JSON format.

		\end{enumerate}
	% subsection prepare_training_data (end)
	\subsection{Training}% (fold)
	\label{sub:training}
		Training:
		\begin{verbatim}
			python train.py -of <OPTIONS_FILE> --log_dir <LOG_DIR>  --name <NAME> --gpus 1
		\end{verbatim}
		\verb+<OPTIONS_FILE>+: JSON file with option overloads. \verb+<LOG_DIR>+: output directory. \verb+<NAME>+: subdirectory for this run. 

		Evaluation:
		\begin{verbatim}
			python test.py <log_directory> --gpu
		\end{verbatim}
		\verb+<log_directory>+: directory containing the checkpoint and options file.
		\begin{verbatim}
			python test.py <log_directory> -tf <TEST_FILE> --gpu
		\end{verbatim}
		\verb+<TEST_FILE>+ will replace the test file in option file.

	% subsection training (end)	
% section spanet (end)	
\section{SPANet for ttbar event}% (fold)
\label{sec:spanet_for_ttbar_event}
	The training and testing datasets from here: \href{http://mlphysics.ics.uci.edu/data/2021_ttbar/}{Link}

	For full ttbar event:
	\begin{itemize}
		\item Training sample:
		\begin{itemize}
			\item Total sample size: 10,009,520
			\item 2t sample size: 2,967,955
		\end{itemize}
		\item Testing sample size: 
			\begin{itemize}
				\item Total sample size: 358,946
				\item 2t sample size: 116,342
			\end{itemize}
	\end{itemize}

	Figure \ref{fig:full_ttbar_result} is the training results for full ttbar events. 
	\begin{figure}[htpb]
		\centering
		\includegraphics[width=0.8\textwidth]{full_ttbar_training_result.png}
		\caption{The training result for full ttbar events.}
		\label{fig:full_ttbar_result}
	\end{figure}
	The results are the same as the numbers given in the SPANet paper.

% section spanet_for_ttbar_event (end)
\section{SPANet for di-Higgs event}% (fold)
\label{sec:spanet_for_dihiggs_event}
	Generate di-Higgs events in MadGraph by following commands:
	\begin{verbatim}
		import model 2HDMtII_NLO
		generate p p > h2 [QCD] QED<=99 QCD<=99	
	\end{verbatim}
	then use the MadSpin let h2 decay to h1h1 and h1 decay to b$\overline{\text{b}}$. The mass of h2 is $\text{1000 GeV}$.

	The \verb+.ini+ file for di-Higgs event (h > b $\overline{\text{b}}$)
	\begin{verbatim}
		[SOURCE]
		mass = log_normalize
		pt = log_normalize
		eta = normalize
		phi = normalize

		[EVENT]
		particles = (h1, h2)
		permutations = [(h1, h2)]

		[h1]
		jets = (b1, b2)
		permutations = [(b1, b2)]

		[h2]
		jets = (b1, b2)
		permutations = [(b1, b2)]			
	\end{verbatim}

	There are three types of events, 0h, 1h, 2h. The number means how many identifiable Higgs in a event.

	Definition of some parameters:
	\begin{itemize}
		\item 	Event proportion:
		\begin{equation}
			\text{Event proportion} \equiv \frac{\text{number of some type events with $i$ jets}}{\text{number of events with $i$ jets}}
		\end{equation}

		\item Jet proportion:
		\begin{equation}
			\text{Jet proportion} \equiv \frac{\text{number of events with $i$ jets}}{\text{total number of events}}
		\end{equation}
		
		\item Event purity:
		\begin{equation}
			\epsilon^{\text{event}} \equiv \frac{\text{number of some type events with and all Higgs are correctly identified}}{\text{number of some type events}} 
		\end{equation}

		\item H purity:
		\begin{equation}
			\epsilon^{\text{h}} \equiv \frac{\text{number of correctly identified Higgs in some type events}}{\text{number of identifiable Higgs in some type events}} 
		\end{equation}
	\end{itemize}

	\subsection{Training sample for di-Higgs events}% (fold)
	\label{sub:training_sample_for_dihiggs_events}
		Di-Higgs events pp->h2, h2->hh, h->bb was generated at $\sqrt{s}= \text{14 TeV}$ using MadGraph. Then pass these events to Pythia8 for showering and hadronization. Then pass to Delphes for detector simulation.
	
		Jets are reconstructed by the anti-kT algorithm with radius $R=0.5$ and are required to have $p_\text{T}\ge \text{20 GeV}$.

		Preselection requirements: $\ge 4$ jets in  $\abs{\eta}<2.5$.

		Define the correct jet assignments by matching them to the simulated truth quarks within an angular distance of $\Delta R = \sqrt{\Delta\eta^2 + \Delta\phi^2}<0.4$.
	% subsection training_sample_for_dihiggs_events (end)

	\subsection{Training result for di-Higgs event}% (fold)
	\label{sub:training_result_for_dihiggs_event}
		For 100k di-Higgs event with $m_{\text{h}_2} = \text{1000 GeV}$:
		\begin{itemize}
			\item Training sample:
			\begin{itemize}
				\item Total sample size: 78,785
				\item 2h sample size: 36,765
				\item 5\% used on validation
			\end{itemize}
			\item Testing sample: 
				\begin{itemize}
					\item Total sample size: 8,753
					\item 2h sample size: 4,130
				\end{itemize}
		\end{itemize}
		\begin{figure}[htpb]
			\centering
			\includegraphics[width=0.8\textwidth]{100k_diHiggs.png}
			\caption{The training result for 100k di-Higgs events.}
			\label{fig:100k_diHiggs_result}
		\end{figure}
		Figure \ref{fig:100k_diHiggs_result} is the training results for 100k di-Higgs events. For 2h events, $\epsilon^{\text{event}} = 0.871, \epsilon^{\text{h}} = 0.933$.

		For 1M di-Higgs event with $m_{\text{h}_2} = \text{1000 GeV}$:
		\begin{itemize}
			\item Training sample:
			\begin{itemize}
				\item Total sample size: 788,160
				\item 2h sample size: 364,773
				\item 5\% used on validation
			\end{itemize}
			\item Testing sample: 
				\begin{itemize}
					\item Total sample size: 87,702
					\item 2h sample size: 40,695
				\end{itemize}
		\end{itemize}
		\begin{figure}[htpb]
			\centering
			\includegraphics[width=0.8\textwidth]{1M_diHiggs.png}
			\caption{The training result for 1M di-Higgs events.}
			\label{fig:1M_diHiggs_result}
		\end{figure}

		Figure \ref{fig:1M_diHiggs_result} is the training results for 1M di-Higgs events. For 2h events, $\epsilon^{\text{event}} = 0.914, \epsilon^{\text{h}} = 0.955$.

		For di-Higgs event with $m_{\text{h}_2} = \text{500 GeV}$:
		\begin{itemize}
			\item Training sample:
			\begin{itemize}
				\item Total sample size: 764,676
				\item 2h sample size: 396,588
				\item 5\% used on validation
			\end{itemize}
			\item Testing sample: 
				\begin{itemize}
					\item Total sample size: 84,687
					\item 2h sample size: 43,818
				\end{itemize}
		\end{itemize}
		\begin{figure}[htpb]
			\centering
			\includegraphics[width=0.8\textwidth]{1M_diHiggs_500_GeV.png}
			\caption{The training result for di-Higgs events with $m_{\text{h}_2} = \text{500 GeV}$.}
			\label{fig:1M_500_GeV_diHiggs_result}
		\end{figure}

		Figure \ref{fig:1M_500_GeV_diHiggs_result} is the training results for $\text{500 GeV}$ di-Higgs events. For 2h events, $\epsilon^{\text{event}} = 0.779, \epsilon^{\text{h}} = 0.871$.

	% subsection training_result_for_dihiggs_event (end)

	\subsection{Apply the pre-trained model on different \texorpdfstring{$m_{\text{h}_2}$}{mh2} sample}% (fold)
	\label{sub:apply_the_pre_trained_model_on_different_mh2_sample}
		The SPANet models in Sec.\ref{sub:training_result_for_dihiggs_event} are trained on $m_{\text{h}_2} = \text{500 GeV}$ and $m_{\text{h}_2} = \text{1000 GeV}$. Test these models on samples with $m_{\text{h}_2} = \text{500 GeV}$, $ \text{750 GeV}$, $\text{1000 GeV}$.

		The results are summarized in Table \ref{tab:SPANet_test_on_different_h2_mass}.
		\begin{table}[htpb]
			\centering
			\caption{SPANet models test on different $m_{\text{h}_2}$ samples.}
			\label{tab:SPANet_test_on_different_h2_mass}
			\begin{tabular}{cccc}
				Training $m_{\text{h}_2}$ (GeV) & Testing $m_{\text{h}_2}$ (GeV) & Event purity & H purity \\
				\hline 
				500      & 500     & 0.779        & 0.871    \\
				500      & 750     & 0.408        & 0.672    \\
				500      & 1000    &  0.390      &  0.665    \\
				1000     & 500    & 0.383        & 0.638   \\
				1000     & 750    & 0.731        & 0.860   \\
				1000     & 1000    & 0.914        & 0.955   
			\end{tabular}
		\end{table}
	% subsection apply_the_pre_trained_model_on_different_mh2_sample (end)

% section spanet_for_dihiggs_event (end)
\section{SPANet for tri-Higgs event}% (fold)
\label{sec:spanet_for_trihiggs_event}
	The tri-Higgs events are signal events (gg->hhh) generated in previous.

	\subsection{Training sample for tri-Higgs event}% (fold)
	\label{sub:training_sample_for_trihiggs_event}
		TriHiggs events gg->hhh was generated at $\sqrt{s} = \text{14 TeV}$ using MadGraph. Then pass these events to Pythia8 for showering and hadronization. In Pythia8, the Higgs are forced to decay to b$\overline{\text{b}}$. Then pass to Delphes for detector simulation.
	
		Jets are reconstructed by the anti-kT algorithm with radius $R=0.5$ and are required to have $p_\text{T}\ge \text{20 GeV}$.

		Preselection requirements: $\ge 6$ jets in  $\abs{\eta}<2.5$.

		Define the correct jet assignments by matching them to the simulated truth quarks within an angular distance of $\Delta R = \sqrt{\Delta\eta^2 + \Delta\phi^2}<0.4$.
	% subsection training_sample_for_dihiggs_event (end)

	\subsection{Training result for tri-Higgs event}% (fold)
	\label{sub:training_result_for_trihiggs_event}
		For 1M tri-Higgs event:
		\begin{itemize}
			\item Training sample:
			\begin{itemize}
				\item Total sample size: 542,681
				\item 3h sample size: 104,467
				\item 5\% used on validation
			\end{itemize}
			\item Testing sample: 
				\begin{itemize}
					\item Total sample size: 60,173
					\item 3h sample size: 11,598
				\end{itemize}
		\end{itemize}
		\begin{figure}[htpb]
			\centering
			\includegraphics[width=0.8\textwidth]{1M_triHiggs.png}
			\caption{The training result for 1M tri-Higgs events.}
			\label{fig:1M_triHiggs_result}
		\end{figure}

		Figure \ref{fig:1M_triHiggs_result} is the training results for 1M tri-Higgs events. For 3h events, $\epsilon^{\text{event}} = 0.344, \epsilon^{\text{h}} = 0.538$.

	% subsection training_result_for_trihiggs_event (end)	

% section spanet_for_trihiggs_event (end)		

\section{\texorpdfstring{$\chi^2$}{chi square} method}% (fold)
\label{sec:chi_2_method}
	\subsection{Di-Higgs}% (fold)
	\label{sub:di_higgs}
		Try all possible combination of final jets and find the minimal mass difference between SM Higgs mass, i.e. minimize this
		\begin{equation}
			\chi^2 = [M(j_1j_2)-M_\text{H}]^2 + [M(j_3j_4)-M_\text{H}]^2
		\end{equation}
		where $M(j_ij_j)$ means the invariant mass of jet $i$ and jet $j$.

		The results is presented in Table \ref{tab:comparison_SPANet_and_chi2}.

		\begin{table}[htpb]
			\centering
			\caption{Performance comparison for di-Higgs events.}
			\label{tab:comparison_SPANet_and_chi2}
			\begin{tabular}{c|c|cc|cc}
					  & Event    & \multicolumn{2}{|c|}{SPANet Efficiency} & \multicolumn{2}{|c}{ $\chi^2$ Efficiency} \\
				$N_\text{Jet}$ & Fraction & Event             & Higgs             & Event            & Higgs           \\
				\hline
				$=4$	  &   0.224       &     1.000       &   1.000       &   1.000           &    1.000             \\
				$=5$	  &   0.334       &     0.950       &   0.975       &   0.675           &    0.737            \\
				$\ge 6$	  &   0.442       &     0.844       &   0.918		&   0.313           &    0.441            \\
				Total	  &   1.000       &     0.914       &   0.955       &   0.582           &    0.665            
			\end{tabular}
		\end{table}

	% subsection di_higgs (end)
		
% section chi_2_method (end)
\end{document} 
